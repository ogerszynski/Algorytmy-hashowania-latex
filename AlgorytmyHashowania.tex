\documentclass[12pt]{article}
\usepackage[utf8]{inputenc}
\usepackage{hyperref}

\title{Algorytmy hashowania – mechanizmy i zastosowania}
\author{Olgierd Gerszyński, indeks 73715}
\date{30 kwietnia 2024}

\begin{document}

\maketitle

\section*{Wstęp}
Algorytmy hashowania pełnią kluczową rolę w ochronie danych w erze cyfrowej, przekształcając dowolne dane wejściowe w unikalny, stałodługościowy ciąg bitów zwanym skrótem, który jest praktycznie niemożliwy do odwrócenia. Te algorytmy zapewniają zarówno szybkość przetwarzania danych, jak i ich bezpieczeństwo, co czyni je niezastąpionymi narzędziami w systemach zabezpieczeń, bazach danych oraz w aplikacjach internetowych.

\section*{Problem i zastosowanie algorytmów hashowania}
Algorytmy hashowania rozwiązują kilka fundamentalnych problemów w dziedzinie bezpieczeństwa danych i szybkiego dostępu do informacji: bezpieczne przechowywanie haseł, weryfikacja integralności danych oraz optymalizacja wyszukiwania.

\section*{Omówienie pięciu algorytmów hashowania}
\subsection*{MD5}
\subsection*{SHA-1}
\subsection*{SHA-256}
\subsection*{B-Crypt}
\subsection*{SHA-3}

\section*{Przykładowe implementacje}
Przykład implementacji algorytmu SHA-256 w języku Python:
\begin{verbatim}
import hashlib

def hash_input(input_string):
    return hashlib.sha256(input_string.encode()).hexdigest()

example_string = "Testowy string"
hashed_string = hash_input(example_string)
print("SHA-256:", hashed_string)
\end{verbatim}

\section*{Kalkulacja złożoności obliczeniowej}
Algorytmy hashowania są zaprojektowane tak, aby były szybkie w generowaniu skrótu, ale jednocześnie wolne i kosztowne obliczeniowo w przypadku prób odwrócenia.

\section*{Wnioski}
Algorytmy hashowania stanowią fundament bezpieczeństwa cyfrowego, umożliwiając ochronę danych przed nieautoryzowanym dostępem oraz ich szybką weryfikację.

\section*{Referencje}
\begin{itemize}
    \item \href{https://sii.pl/blog/metody-atakow-oraz-podstawowe-algorytmy-szyfrowania-w-cyberbezpieczenstwie/}{Metody ataków oraz podstawowe algorytmy szyfrowania w cyberbezpieczeństwie}
    \item \href{https://pl.wikipedia.org/wiki/Funkcja_skr%C3%B3tu}{Funkcja skrótu – Wikipedia}
    \item \href{https://academy.binance.com/pl/articles/what-is-hashing}{What is Hashing? – Binance Academy}
    \item \href{https://klinikadanych.pl/artykuly/rodzaje-algorytmow-szyfrujacych-oraz-ich-implementacje-programowe}{Rodzaje algorytmów szyfrujących oraz ich implementacje programowe}
    \item \href{https://ssamolej.kia.prz.edu.pl/dydaktyka/KiBD/03_KiBD_szyfry_blokowe.pdf}{Szyfry blokowe - PDF}
    \item \href{https://informatyka.2ap.pl/ftp/3d/algorytmy/podr%C4%99cznik_algorytmy.pdf}{Podręcznik algorytmy - PDF}
\end{itemize}

\end{document}
